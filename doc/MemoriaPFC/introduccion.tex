\section{Contexto y motivación}

Innovar y buscar nuevos métodos para la enseñanza es un reto para la comunidad universitaria,
sea el campo que sea. Los nuevos \textit{estándares} de enseñanza obligan al profesorado a buscar
alternativas para transmitir conocimientos a sus alumnos.\\

En el campo de las enseñanzas técnicas, y concretamente en el de la informática, quizás pueda
ser una buena idea cambiar la metodología, no tanto de trabajo, sino el enfoque de desarrollo e 
intentar motivar al alumnado con problemas más realistas, más aplicables a un entorno real (ó 
relativamente real).\\

En el caso que nos ocupa, los sistemas expertos, es fácil proponerle al estudiante un problema en 
el que diseñe un sistema que deduzca parentescos familiares, por poner un ejemplo. Sin embargo, 
a un programador normalmente le gusta hacer cosas más visibles, más útiles, y si puede ser, más
entretenidas. Ese es el campo de acción de este proyecto\\

\textbf{Resistencia en Cádiz: 1812} pretende aportar un grano de arena en ese aspecto, otorgando
un entorno de desarrollo usable, ameno y productivo. De esta forma, se desarrollará la aplicación
para que pueda ser utilizada en un marco de aprendizaje concreto, y hacer posible al alumnado
conocer las características de un \textbf{sistema experto basado en reglas}, diseñar el suyo
propio y hacerlo competir con el del resto de los compañeros.

\section{Contenido del documento}