\section{Introducción del proyecto}

El documento que tiene usted en sus manos, es una memoria de todo el proceso de desarrollo
del proyecto \textbf{Ampliación y reingeniería de un sistema experto basado en reglas con
fines educativos}, llamado extra-oficialmente \textbf{Resistencia en Cádiz: 1812}.\\

\section{Objetivos}

Probablemente lo primero que hay que definir en cualquiero proyecto (sea de 
software ó de otro campo), son los objetivos que queremos cumplir con dicho proyecto.
No tiene demasiado sentido recorrer un camino si no sabemos hacia donde queremos ir 
exactamente.\\

En el caso del proyecto que nos ocupa, \textbf{Resistencia en Cádiz: 1812} ha sido
un proceso bastante importante, el cual había que dejar claro desde el comienzo, ya
que este proyecto puede expandirse. Además el hecho que sea una \textit{evolución}
de proyectos anteriores, era un indicativo más de donde se quería incidir.

Era necesario también una buena definición de los objetivos para ``no perder el norte'',
y poder recurrir a ellos en caso de un bloqueo ó problema durante el desarrollo del
mismo.\\

Así mismo, es importante separar requisitos funcionales de requisitos transversales.
Con los primeros me refiero (obviamente) a las funcionalidades ó posibilidades de las
que dispondrá el usuario de este proyecto una vez terminado. En el caso de los
requisitos transversales, se podrían considerar aquellos requisitos que, si bien no
son funciones del proyecto, si que influyen en la calidad del mismo, tanto en el
uso como en el mantenimiento futuro.\\

\noindent Los objetivos funcionales que se marcaron en un principio fueron los siguientes:

\begin{itemize}
\item Proporcionar una interfaz amigable para la competición y prueba
  de distintos sistemas expertos realizados para la aplicación \textbf{La batalla
  del Guadalete}
\item Ofrecer un entorno cómodo para la gestión de equipos y estrategias.
\item Generar una \textit{biblioteca} de partidas jugadas, de forma que no perdamos
  ninguna partida simulada en el sistema
\item Implementar competiciones entre sistemas expertos (ligas, copas...)
\item Realizar un entorno de pruebas: simulaciones rápidas, humano contra un sistema
  experto.
\end{itemize}

\noindent Por otro lado, los requisitos transversales a cumplir por el proyecto serían:

\begin{itemize}
\item \textbf{Internacionalización} de la aplicación.
\item Diseño coherente para facilitar el mantenimiento y ampliación de la aplicación.
\item Seguir la guia de estilo de codificación en Python \cite{web:pep8}
\item Código escrito en inglés, para una mejor adaptación de futuros colaboradores
\item Código comentado siguiendo un formato concreto, para poder usar herramientas de 
generación automática de documentación.
\item Uso de algún \textbf{Sistema de Control de Versiones} (SCM) para no tener problemas
de copias de seguridad.
%\item Instalaciones automáticas para plataformas Linux y Windows.
\end{itemize}

Al final del proyecto, volveremos sobre estos objetivos para ver si los hemos cumplido,
hemos modificado alguno durante el desarrollo, ó si alguno se ha quedado ``descolgado''.
